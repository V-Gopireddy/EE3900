\documentclass[journal,12pt,twocolumn]{IEEEtran}

\usepackage{setspace}
\usepackage{gensymb}

\singlespacing


\usepackage[cmex10]{amsmath}

\usepackage{amsthm}

\usepackage{mathrsfs}
\usepackage{txfonts}
\usepackage{stfloats}
\usepackage{bm}
\usepackage{cite}
\usepackage{cases}
\usepackage{subfig}

\usepackage{longtable}
\usepackage{multirow}

\usepackage{enumitem}
\usepackage{mathtools}
\usepackage{steinmetz}
\usepackage{tikz}
\usepackage{circuitikz}
\usepackage{verbatim}
\usepackage{tfrupee}
\usepackage[breaklinks=true]{hyperref}
\usepackage{graphicx}
\usepackage{tkz-euclide}
\usepackage{float}

\usetikzlibrary{calc,math}
\usepackage{listings}
    \usepackage{color}                                            %%
    \usepackage{array}                                            %%
    \usepackage{longtable}                                        %%
    \usepackage{calc}                                             %%
    \usepackage{multirow}                                         %%
    \usepackage{hhline}                                           %%
    \usepackage{ifthen}                                           %%
    \usepackage{lscape}     
\usepackage{multicol}
\usepackage{chngcntr}

\DeclareMathOperator*{\Res}{Res}

\renewcommand\thesection{\arabic{section}}
\renewcommand\thesubsection{\thesection.\arabic{subsection}}
\renewcommand\thesubsubsection{\thesubsection.\arabic{subsubsection}}

\renewcommand\thesectiondis{\arabic{section}}
\renewcommand\thesubsectiondis{\thesectiondis.\arabic{subsection}}
\renewcommand\thesubsubsectiondis{\thesubsectiondis.\arabic{subsubsection}}


\hyphenation{op-tical net-works semi-conduc-tor}
\def\inputGnumericTable{}                                 %%

\lstset{
%language=C,
frame=single, 
breaklines=true,
columns=fullflexible
}
\makeatletter
\setlength{\@fptop}{0pt}
\makeatother
\begin{document}
\newtheorem{theorem}{Theorem}[section]
\newtheorem{problem}{Problem}
\newtheorem{proposition}{Proposition}[section]
\newtheorem{lemma}{Lemma}[section]
\newtheorem{corollary}[theorem]{Corollary}
\newtheorem{example}{Example}[section]
\newtheorem{definition}[problem]{Definition}

\newcommand{\BEQA}{\begin{eqnarray}}
\newcommand{\EEQA}{\end{eqnarray}}
\newcommand{\define}{\stackrel{\triangle}{=}}
\bibliographystyle{IEEEtran}
\providecommand{\mbf}{\mathbf}
\providecommand{\pr}[1]{\ensuremath{\Pr\left(#1\right)}}
\providecommand{\qfunc}[1]{\ensuremath{Q\left(#1\right)}}
\providecommand{\sbrak}[1]{\ensuremath{{}\left[#1\right]}}
\providecommand{\lsbrak}[1]{\ensuremath{{}\left[#1\right.}}
\providecommand{\rsbrak}[1]{\ensuremath{{}\left.#1\right]}}
\providecommand{\brak}[1]{\ensuremath{\left(#1\right)}}
\providecommand{\lbrak}[1]{\ensuremath{\left(#1\right.}}
\providecommand{\rbrak}[1]{\ensuremath{\left.#1\right)}}
\providecommand{\cbrak}[1]{\ensuremath{\left\{#1\right\}}}
\providecommand{\lcbrak}[1]{\ensuremath{\left\{#1\right.}}
\providecommand{\rcbrak}[1]{\ensuremath{\left.#1\right\}}}
\theoremstyle{remark}
\newtheorem{rem}{Remark}
\newcommand{\sgn}{\mathop{\mathrm{sgn}}}
\providecommand{\abs}[1]{\vert#1\vert}
\providecommand{\res}[1]{\Res\displaylimits_{#1}} 
\providecommand{\norm}[1]{\lVert#1\rVert}
%\providecommand{\norm}[1]{\lVert#1\rVert}
\providecommand{\mtx}[1]{\mathbf{#1}}
\providecommand{\mean}[1]{E[ #1 ]}
\providecommand{\fourier}{\overset{\mathcal{F}}{ \rightleftharpoons}}
%\providecommand{\hilbert}{\overset{\mathcal{H}}{ \rightleftharpoons}}
\providecommand{\system}{\overset{\mathcal{H}}{ \longleftrightarrow}}
	%\newcommand{\solution}[2]{\textbf{Solution:}{#1}}
\newcommand{\solution}{\noindent \textbf{Solution: }}
\newcommand{\cosec}{\,\text{cosec}\,}
\providecommand{\dec}[2]{\ensuremath{\overset{#1}{\underset{#2}{\gtrless}}}}
\newcommand{\myvec}[1]{\ensuremath{\begin{pmatrix}#1\end{pmatrix}}}
\newcommand{\mydet}[1]{\ensuremath{\begin{vmatrix}#1\end{vmatrix}}}
\numberwithin{equation}{subsection}
\makeatletter
\@addtoreset{figure}{problem}
\makeatother
\let\StandardTheFigure\thefigure
\let\vec\mathbf
\renewcommand{\thefigure}{\theproblem}
\def\putbox#1#2#3{\makebox[0in][l]{\makebox[#1][l]{}\raisebox{\baselineskip}[0in][0in]{\raisebox{#2}[0in][0in]{#3}}}}
     \def\rightbox#1{\makebox[0in][r]{#1}}
     \def\centbox#1{\makebox[0in]{#1}}
     \def\topbox#1{\raisebox{-\baselineskip}[0in][0in]{#1}}
     \def\midbox#1{\raisebox{-0.5\baselineskip}[0in][0in]{#1}}
\vspace{3cm}
\title{QUIZ-1}
\author{Vojeswitha Gopireddy \\ AI20BTECH11024}
\maketitle
\newpage
\bigskip
\renewcommand{\thefigure}{\theenumi}
\renewcommand{\thetable}{\theenumi}
Download all latex-tikz codes from 
%
\begin{lstlisting}
https://github.com/V-gopireddy/EE3900/blob/main/Quiz1/Quiz-1.tex
\end{lstlisting}
\section{QUESTION 2.19 (d,e,f)}
For each of the following impulse responses of LTI systems, indicate whether or not the system is stable:\\
\begin{enumerate}
    \item $h[n] = \sin(n\pi/3)u[n] $
    \item $h[n] =\brak{3/4}^{\abs{n}}\cos(n\pi/4 +\pi/4)$
    \item $h[n] = 2u[n+5] - u[n] - u[n-5]$
\end{enumerate}
%
\section{SOLUTION}
%
\begin{definition}
We say that a system is \textbf{stable} if it produces a bounded output for every possible bounded input, i.e it satisfies the BIBO(Bounded-input-Bounded-output) condition.
\end{definition}
%
\begin{lemma}
A system with impulse response $h[n]$ is said to be BIBO stable if and only if $h[n]$ is absolutely summable
\begin{align}
    S = \sum_{n=-\infty}^{\infty}\abs{h[n]}<\infty
\end{align}
\end{lemma}
\begin{lemma}
LTI system with impulse response 
\begin{align}
    h[n] = \sin(n\pi/3)u[n]
\end{align}
is unstable
\end{lemma}
\begin{proof}
We have,
\begin{align}
    h[n] &= \sin\brak{\frac{n\pi}{3}}u[n]\\
         &= \begin{cases}
         \sin\brak{\frac{n\pi}{3}}, &  n \geq 0 \\~\\[-1em]
	     0, &  n<0 
         \end{cases}
\end{align}
Therefore
\begin{align}
    S &= \sum_{n=-\infty}^{\infty}\abs{h[n]}\\ 
      &= \sum_{n=0}^{\infty} \abs{\sin\brak{\frac{n\pi}{3}}}\\
      &= \sum_{n=0}^{\infty} \sqrt{3}
      = \infty
\end{align}
Since, 
\begin{align}
    S = \infty 
\end{align}
The system is unstable
\end{proof}
\begin{lemma}
LTI system with impulse response 
\begin{align}
    h[n] =\brak{3/4}^{\abs{n}}\cos(n\pi/4 +\pi/4)
\end{align}
is stable
\end{lemma}
\begin{proof}
Since $-1\leq \cos(n\pi/4 +\pi/4)\leq 1$
\begin{align}
    S &= \sum_{n=-\infty}^{\infty}\abs{h[n]}\\
    &= \sum_{n=-\infty}^{\infty}\abs{\brak{3/4}^{\abs{n}}\cos(n\pi/4 +\pi/4)}\\
    &\leq \sum_{n=-\infty}^{\infty}\abs{\brak{\frac{3}{4}}^{\abs{n}}}\\
    &=1+ 2\sum_{n=1}^{\infty}\abs{\brak{\frac{3}{4}}^{n}}
    = 7
\end{align}
Since,
\begin{align}
    S = 7 < \infty
\end{align}
The system is stable 
\end{proof}
\begin{lemma}
LTI system with impulse response 
\begin{align}
   h[n] = 2u[n+5] - u[n] - u[n-5]
\end{align}
is stable
\end{lemma}
\begin{proof}
We have
\begin{align}
    h[n] &= 2u[n+5] - u[n] - u[n-5]\\
         &= \begin{cases}
         2, & -5 \leq n < 0 \\~\\[-1em]
	     1, & 0 \leq n < 5 \\~\\[-1em]
	     0, & \text{otherwise}
         \end{cases}
\end{align}
Therefore
\begin{align}
    S &= \sum_{n=-\infty}^{\infty}\abs{h[n]}\\ 
    &=  \sum_{n=-5}^{0}\abs{2}+ \sum_{n=0}^{-5}\abs{1}
    = 15
\end{align}
Since,
\begin{align}
    S = 15 < \infty
\end{align}
The system is stable
\end{proof}
\end{document}
